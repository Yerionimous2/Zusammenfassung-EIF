\documentclass{article}

% config.tex
\usepackage{tcolorbox}
\usepackage{amssymb}
\usepackage{amsmath}
\usepackage[ngerman]{babel}
\usepackage[T1]{fontenc}
\usepackage[utf8]{inputenc}
\tcbuselibrary{listings, skins, breakable}

\newcounter{block}[section]
\renewcommand{\theblock}{\thesection.\arabic{block}}

% Automatisches Zurücksetzen von block bei jedem \section
\usepackage{etoolbox}
\preto\section{\setcounter{block}{0}}

\newtcolorbox{block}[1]{%
  enhanced,
  breakable,
  colback=green!5!white,
  colframe=green!65!black,
  coltitle=black,
  fonttitle=\bfseries,
  title=\thesection.\number\numexpr\value{block}+1\relax\ #1,
  before upper={},
  after upper={\stepcounter{block}},
  boxrule=0.8pt,
  arc=0mm,
  left=1mm,
  right=1mm,
  top=1mm,
  bottom=1mm,
}

\newenvironment{beweis}[1][Beweis]{
  \par\noindent\textbf{#1.}\\ 
}{
    \hfill \framebox[1.5ex][c]{}
    \par\vspace{2em}
    \noindent
}

\begin{document}

\begin{titlepage}
    \centering
    \includegraphics[width=0.3\textwidth]{Uni Logo.png} 
    \vspace*{2cm}
    
    {\Large\bfseries Einführung Fachdidaktik Informatik\par}
    \vspace{1.5cm}
    
    \textbf{Peter Minor}\\
    Sommersemester 2025
    
    \vfill    
    \vspace{1.5cm}
    {\large \today\par}
\end{titlepage}

\tableofcontents

\newpage

\section{Didaktische Dimensionen}
Die \textbf{Didaktik} fragt: \emph{Was}, \emph{wie}, \emph{warum}, \emph{wann}, \emph{wo}, \emph{mit wem}, \emph{womit} und \emph{für wen} gelehrt werden soll.

\begin{block}{Definition: Didaktische Dimensionen}
    Didaktische Dimensionen sind grundlegende Perspektiven, unter denen Unterricht geplant und reflektiert wird:
    \begin{itemize}
        \item \textbf{Inhalte}, \textbf{Ziele}, \textbf{Themen}
        \item \textbf{Methodik}, \textbf{Medien}, \textbf{Organisation des Lernens}
        \item \textbf{Bildung} als übergeordnetes Ziel
    \end{itemize}
\end{block}

\begin{block}{Didaktik vs Methodik}
    \textbf{Didaktik} beschäftigt sich mit der Frage \emph{was}, \emph{warum}, \emph{für wen} und \emph{mit welchem Ziel} gelehrt werden soll. \\
    \textbf{Methodik} hingegen fragt \emph{wie}, \emph{mit welchen Mitteln} und \emph{in welcher Form} der Unterricht konkret umgesetzt wird. \\[0.5em]
    \textit{Kurz:} Didaktik = \emph{Inhalts- und Zielperspektive},\\ \quad Methodik = \emph{Umsetzungs- und Prozessperspektive}.
\end{block}


\section{Hintergrund und Modelle}
\begin{block}{Didactica Magna – Comenius (1657)}
    Ziel: ``Die vollständige Kunst, alle Menschen alles zu lehren'' \\
    Leitideen:
    \begin{itemize}
        \item Rasch, angenehm und gründlich lehren
        \item Wahrheiten mit Beispielen aus mechanischen Künsten
        \item Feste Reihenfolge nach Alter, Zeit, Entwicklung
    \end{itemize}
\end{block}

\begin{block}{Allgemeine Didaktik}
    Basierend auf Lerntheorien. Modelle u.a.:
    \begin{itemize}
        \item Bildungstheoretisch / kritisch-konstruktiv
        \item Lerntheoretisch (behavioristisch, kognitivistisch, konstruktivistisch)
        \item Informationstheoretisch-kybernetisch
    \end{itemize}
    Konzepte:
    \begin{itemize}
        \item Kontextorientierung
        \item Forschend-entwickelnder Unterricht
        \item Projektorientierung
    \end{itemize}
    Prinzipien:
    \begin{itemize}
        \item An Grundideen orientieren
        \item Beziehungen herstellen
        \item Adäquat visualisieren
    \end{itemize}
\end{block}

\section{Informatikdidaktik}
\begin{block}{Fachdidaktik ist keine Abbilddidaktik}
    Ziel ist nicht die reine Weitergabe der Fachwissenschaft, sondern:
    \begin{itemize}
        \item Entwicklung von Welt- und Selbstverständnis Jugendlicher fördern
        \item Kooperative Reflexion mit Allgemeiner Didaktik und Bildungstheorie
        \item Fachinhalte auf Lebenswelt und Bildungsziele beziehen
    \end{itemize}
\end{block}

\begin{block}{Bezugswissenschaften}
    \begin{itemize}
        \item Fachwissenschaft Informatik
        \item Psychologie, Soziologie
        \item Allgemeine Didaktik, Bildungstheorie
    \end{itemize}
\end{block}

\section{Wissenschaftliche Perspektive}
\begin{block}{Forschungsdisziplin Didaktik der Informatik}
    \begin{itemize}
        \item Inhaltliche, methodische, mediale Konzepte
        \item Ziel: Qualitätssicherung informatischer Bildung
        \item Veranstaltungen: INFOS, DeLFI, WiPSCE
    \end{itemize}
\end{block}

\section{Normenproblem \& Bildungsbegriff}
\begin{block}{Normenproblem}
    Bildung ist wertgebunden und abhängig von gesellschaftlichen Idealen.\\
    Folge: Didaktische Forschung ist komplex und pluralistisch.
\end{block}

\begin{block}{Bildung}
    Bildung = eigenständige, individuelle Repräsentation von Kultur \\
    Begriffe wie ``Bildung'' oder ``Didaktik'' schwer ins Englische übersetzbar, da sie geisteswissenschaftlich tief verwurzelt sind.
\end{block}

\section{Gestaltung von Unterricht}
\begin{block}{Methoden in Vorlesungen (nach Weickert)}
    \textbf{Starter:} Kennenlernspiele, lebendige Statistik \\
    \textbf{Begleiter:} Mitdenken anregen, Brainstorming, Lernstopp \\
    \textbf{Evaluierer:} Fragebogen, Blitzlicht, ``Heute habe ich gelernt, ...''
\end{block}

\section{Lernumgebungen (Beispiele)}
\begin{itemize}
    \item \textbf{BlueJ}: Objektorientiertes Programmieren visuell erleben
    \item \textbf{Kara}: Steuerung eines Marienkäfers über Programmierung
    \item \textbf{PuMa}: Puppenhaus-Automation als niederschwellige Einführung in Programmierlogik
\end{itemize}

\section{Zusammenfassung: Prüfungsrelevantes Wissen Teil A}

\begin{itemize}
    \item Definition und Bedeutung didaktischer Dimensionen kennen
    \item Unterschied zwischen Allgemeiner Didaktik und Fachdidaktik verstehen
    \item Relevante Didaktik-Modelle benennen können
    \item Fachdidaktik Informatik als eigenständige Disziplin einordnen
    \item Bildungsbegriff und Normenproblem reflektieren
    \item Methodenvielfalt im Unterricht / in Vorlesungen erläutern
    \item Beispiele für Lernumgebungen im Informatikunterricht kennen
\end{itemize}

\section{Informatische Bildung}
\begin{block}{Allgemeinbildung und Schule}
    \begin{itemize}
        \item \textbf{Bildungsauftrag der Schule} nach Fend (1980): Qualifikation, Selektion, Sozialisation, Legitimation
        \item Allgemeinbildung als Vorbereitung auf Beruf, Studium und mündige Teilhabe an Gesellschaft
        \item Comenius: „Nur der gebildete Mensch ist Mensch“
        \item Klafki: \textbf{Epochaltypische Schlüsselprobleme} als Maßstab für relevante Bildungsinhalte
    \end{itemize}
\end{block}

\begin{block}{Beitrag der Informatik zur Allgemeinbildung}
    \begin{itemize}
        \item \textbf{IU als einziges Fach} mit technisch-naturwissenschaftlichem Fokus (Modrow 2005)
        \item Vermittlung von \textbf{Kernkonzepten der Informatik} ist zentral (Hartmann/Nievergelt)
        \item IU leistet Beitrag zur digitalen Mündigkeit und zur Reflexion gesellschaftlicher Entwicklungen
    \end{itemize}
\end{block}

\begin{block}{Problem der Realisierung}
    \begin{itemize}
        \item Zieltrias (Hartmann): Alltagsrelevanz, wissenschaftliches Verständnis, gesellschaftliche Reflexion
        \item Informatik gelingt diese Verknüpfung bislang unzureichend
    \end{itemize}
\end{block}

\section{Orientierungen und Konzepte}
\begin{block}{GI-Gesamtkonzept Informatische Bildung (2000)}
    \begin{itemize}
        \item Bildung durch Erschließen von Grundlagen, Methoden, Anwendungen und gesellschaftlicher Bedeutung von IS
        \item \textbf{Bewusstes Thematisieren} von Informatik erforderlich — keine bloße Techniknutzung
        \item \textbf{Abgrenzung} zur ITG (bedienorientiert) und Medienpädagogik
    \end{itemize}
\end{block}

\begin{block}{Vier Leitlinien (GI 2000)}
    \begin{enumerate}
        \item Interaktion mit Informatiksystemen
        \item Wirkprinzipien von IS
        \item Informatische Modellierung
        \item Wechselwirkungen: IS, Individuum, Gesellschaft
    \end{enumerate}
\end{block}

\section{Bildungsstandards und Schulstufen}

\begin{block}{Schwerpunkte je Schulstufe}
    \begin{itemize}
        \item \textbf{Primarstufe:} Werkzeuge, Grundkenntnisse, digitale Spaltung vermeiden
        \item \textbf{Sek I:} Handlungskompetenz, Systematisierung von Alltagserfahrungen
        \item \textbf{Sek II:} formale Methoden, informatisches Modellieren
    \end{itemize}
\end{block}

\begin{block}{Bildungsstandards (GI 2008 und KMK 2004)}
    \begin{itemize}
        \item Ergebnisorientierung (Kompetenzmodell nach Weinert)
        \item Drei Niveaustufen: Mindest-, Regel-, Maximalstandards
        \item Intention: \textbf{alle SuS} sollen IT zum Nutzen bewältigen können
    \end{itemize}
\end{block}

\section{Beruf und Wissenschaftspropädeutik}

\begin{block}{IU und Berufswelt}
    \begin{itemize}
        \item Förderung kreativen, algorithmischen Denkens, Transferfähigkeit, Teamarbeit
        \item Berufliche Orientierung durch technische Erfahrungen in der Schule
    \end{itemize}
\end{block}

\begin{block}{Wissenschaftspropädeutik}
    \begin{itemize}
        \item Aneignung von Grundlagenwissen, Reflexionsfähigkeit, Lernstrategien
        \item Informatik als Zugang zu ingenieurwissenschaftlichem Denken
    \end{itemize}
\end{block}

\section{Internationale Perspektiven}
\begin{block}{UNESCO ICT Curriculum (2000)}
    \begin{enumerate}
        \item ICT Literacy (Computer bedienen)
        \item ICT in Fächern anwenden
        \item ICT fachübergreifend integrieren
        \item ICT-Spezialisierung
    \end{enumerate}
\end{block}

\begin{block}{Being Fluent with IT (NRC 1999)}
    \begin{itemize}
        \item Literacy (Fakten), Capabilities (Fähigkeiten), Concepts (Konzepte)
        \item Ziel: dauerhafte, tiefgreifende IT-Kompetenz, nicht nur Bedienung
    \end{itemize}
\end{block}

\section{Digitale Mündigkeit und aktuelle Entwicklungen}

\begin{block}{Digitale Mündigkeit}
    \begin{itemize}
        \item Kritische Reflexion, Urteilskompetenz, gesellschaftliche Verantwortung
        \item Kompetenzrahmen: Problemlösen, Automatisierung, Algorithmisches Denken
    \end{itemize}
\end{block}

\begin{block}{Rahmen und Strategien (Auswahl)}
    \begin{itemize}
        \item Medienkompetenzrahmen NRW (2018–)
        \item KMK-Strategie „Bildung in der digitalen Welt“ (2016–)
        \item EU DigComp 2.2 (2022), UNESCO Framework (2018)
    \end{itemize}
\end{block}

\section{Zusammenfassung: Prüfungsrelevantes Wissen Teil B}

\begin{itemize}
    \item Begriff und Zielsetzung informatischer Bildung definieren können
    \item Beitrag der Informatik zur Allgemeinbildung (z.B. Klafki, digitale Mündigkeit) benennen
    \item Vier Leitlinien des GI-Gesamtkonzepts (2000) kennen und erläutern
    \item Unterschiede zwischen Informatik, ITG, Medienbildung und ICT Literacy verstehen
    \item Bildungspolitische Rahmenwerke kennen (KMK, UNESCO, Medienkompetenzrahmen NRW, EU DigComp)
    \item Struktur und Inhalte der Bildungsstandards Informatik (GI 2008, KMK) benennen
    \item Bedeutung von Wissenschaftspropädeutik und Berufsvorbereitung im Informatikunterricht reflektieren
    \item Internationale Curricula (z.B. UNESCO ICT, ACM K12, FIT-Konzept) einordnen können
    \item Herausforderungen und aktuelle Probleme des Informatikunterrichts (v.a. in NRW) benennen
\end{itemize}

\section{Thema C: Was ist Informatik}
\begin{block}{Definition und Abgrenzung}
    \begin{itemize}
        \item \textbf{Informatik} beschäftigt sich mit der Darstellung, Speicherung, Übertragung und Verarbeitung von Information.
        \item Die Fragestellungen und Inhalte der Fachwissenschaft Informatik unterscheiden sich von populären Vorstellungen (z.B. Office-Anwendungen, reine Mediennutzung oder Elektrotechnik zählen nicht zur Informatik).
        \item Informatik ist sowohl eine \textbf{Grundlagenwissenschaft} als auch eine \textbf{Ingenieurwissenschaft}.
        \item Informatik betrachtet Information aus verschiedenen Perspektiven: technisch, personal, organisationsbezogen und medial.
    \end{itemize}
\end{block}

\begin{block}{Was gehört (nicht) zur Informatik?}
    \centering
    \begin{tabular}{|p{5cm}|p{5cm}|}
        \hline
        \textbf{Das ist Informatik} & \textbf{Das ist keine Informatik} \\
        \hline
        Algorithmisches Denken & Office-Handhabung \\
        Programmieren & Elektrotechnik \\
        Hardware und Software & Digitale Medienkunde \\
        Theoretische Informatik & Internetanwendungen \\
        Datenmanagement & Wie baue ich einen PC \\
        Netzwerke & Homepage-Design \\
        Informationsverarbeitung & Toaster \\
        Datensicherheit & \\
        Informatik und Gesellschaft & \\
        \hline
    \end{tabular}
\end{block}

\begin{block}{Historische Entwicklung}
    \begin{itemize}
        \item \textbf{Charles Babbage}: Difference Engine und Analytical Engine als erste Konzepte universeller Maschinen.
        \item \textbf{Konrad Zuse}: Erste programmierbare Rechner (Z1, Z3), Entwicklung des Plankalküls als früher Programmiersprache.
        \item Entwicklung von mechanischen und elektromechanischen Rechnern (z.B. MARK I, ENIAC) zur von-Neumann-Architektur und modernen Computern.
        \item Entdeckung des Transistors und Miniaturisierung ermöglichen Mikroprozessoren und heutige Computertechnik.
        \item Vernetzung von Rechnern (z.B. ARPANET, später Internet) und Entwicklung von Software prägen die Informatik maßgeblich.
    \end{itemize}
\end{block}

\begin{block}{Theoretische Grundlagen}
    \begin{itemize}
        \item \textbf{Formale Logik} bildet die Grundlage der Informatik (von Aristoteles bis zur modernen Logik).
        \item \textbf{Kalkül} (Leibniz) und \textbf{Algorithmus} (Turing) als zentrale Konzepte:
        \begin{itemize}
            \item Allgemeinheit, Endlichkeit, Determiniertheit, Terminierung, Determinismus
        \end{itemize}
        \item Der \textbf{Gödelsche Unvollständigkeitssatz} zeigt die Grenzen formaler Systeme auf.
        \item Die \textbf{Turingmaschine} dient als Modell für Berechenbarkeit und Algorithmik.
    \end{itemize}
\end{block}

\begin{block}{Informatik und Gesellschaft}
    \begin{itemize}
        \item Informatik prägt Berufswelt, Kommunikation und gesellschaftliche Strukturen grundlegend.
        \item Digitale Mündigkeit und kritische Reflexion sind wichtige Ziele informatischer Bildung.
        \item Informatik ist interdisziplinär mit Psychologie, Soziologie und Didaktik verbunden.
        \item Das sogenannte \textbf{Normenproblem}: Bildung ist wertgebunden und von gesellschaftlichen Idealen geprägt, was die didaktische Forschung komplex und pluralistisch macht.
    \end{itemize}
\end{block}

\section{Zusammenfassung: Prüfungsrelevantes Wissen Teil C}
\begin{itemize}
    \item Definition und Abgrenzung der Informatik kennen und erläutern können
    \item Historische Entwicklungsschritte und zentrale Persönlichkeiten benennen
    \item Theoretische Grundlagen (Logik, Algorithmus, Turingmaschine) verstehen
    \item Wechselwirkungen zwischen Informatik, Gesellschaft und Bildung reflektieren können
\end{itemize}

\section{Elemente der Unterrichtsgestaltung}
\begin{block}{Unterricht}
    \begin{itemize}
        \item Gezielte, geplante Vermittlung von Wissen, Fähigkeiten und praktischem können
        \item Keine zufälligen Belehrungen oder Hinweise
        \item Kennzeichen von Unterricht an Schulen:
        \begin{itemize}
            \item Pädagogische Gerichtetheit
            \item Planmäßigkeit
            \item Institutionalisierung
            \item Verberuflichung
        \end{itemize}
    \end{itemize}
\end{block}

Unterricht ohne Ziel: Diffus

\begin{block}{Lernziele}
    Lernzieldimensionen:
    \begin{itemize}
        \item Kognitive Lernziele
        \begin{itemize}
            \item Wissen, Verstehen, Anwenden, Analysieren, Synthesieren, Bewerten
        \end{itemize}
        \item Affektive Lernziele
        \begin{itemize}
            \item Beobachten, Beantworten, Bewerten, ..., Weltanschauung
        \end{itemize}
        \item Psychomotorische Lernziele
        \begin{itemize}
            \item Imitatieren, Manipulieren, Präzisieren, ..., Verinnerlichung
        \end{itemize}
    \end{itemize}
\end{block}

\begin{block}{AFBs}
    \textbf{Allgemeine Fachliche Begriffe} (AFBs) sind zentrale Begriffe der Informatik, die in der Schule vermittelt werden sollen. Sie dienen als Grundlage für die Entwicklung von Kompetenzen und Fähigkeiten im Umgang mit informatischen Systemen.
    \begin{itemize}
        \item AFB I: Wissen wiedergeben, Methoden anwenden (30-40\%)
        \item AFB II: Probleme lösen, Konzepte verstehen (50-60\%)
        \item AFB III: Systeme analysieren, kritisch reflektieren (10-20\%)
    \end{itemize}
\end{block}

\begin{block}{Kompetenzmodell}
    \begin{itemize}
        \item Fähigkeit, persönliches, berufliches und gesellschaftliches Leben zu führen
        \item Aufgeteilt in:
        \begin{itemize}
            \item Sachkompetenz: Kentnisse und Einsichten
            \item Sozialkompetenz: Fähigkeit, eigene Ziele im Einklang mti anderen Beteiligten zu verfolgen
            \item Methodenkompetenz: Fähigkeit, eigenen Lernprozess zu gestalten
            \item Personale Kompetenz: Einstellungen, Motivationen, die das Handeln beeinflussen(Selbstvertrauen)
        \end{itemize}
    \end{itemize}
\end{block}

\begin{block}{Bildungsstandards}
    Kamen durch Pisa 2000, KMK wollte einheitl. Standards
    \begin{itemize}
        \item Kompetenzorientierung anstatt INPUT-Orientierung
        \item Mindeststandards für alle Schüler
        \item Regelstandards für die meisten Schüler
        \item Maximalstandards für die leistungsstärksten Schüler
    \end{itemize}
    Kompetenzorientierung theoretisch zwar gut, aber meist bleiben die Inhalte die selben
\end{block}

\begin{block}{Gegenstand - Inhalt - Thema}
    Thema benennt einen Inhalt, der an einem Gegenstand vermittelt wird.
    Beispiele für Inhaltsbereiche aus der Informatik:
    \begin{itemize}
        \item DB und Informationssysteme
        \item Rechnerarchitektur
        \item Geschichte der Informatik
        \item Sprach- und Signalverarbeitung
    \end{itemize}
    Beispielprozess für Themenfindung:
    \begin{center}
        \begin{tabular}{|p{5cm}|p{5cm}|}
            \hline
            \textbf{Prozess zur Themenfindung} & \textbf{Beispiel} \\
            \hline
            Idee: Thema wird grob formuliert & Verarbeitung von Bildern mit Informatiksystemen \\
            \hline
            Was soll mit dem Thema vermittelt werden? & \\
            \hline
            Lernziele & Erfahren, dass die Verarbeitung von grafischen Daten zur Veränderung der Informationen führen kann. \\
            \hline
            Lerninhalte & Flussdiagramm zur Notation von Algorithmen \\
            \hline
            Fächerverbindungen & Computerkunst, Optik \\
            \hline
            Wie können die Lerninhalte interessant vermittelt werden? & \\
            \hline
            Alltagsbezüge & Computergrafik \\
            \hline
            Medien & Paint \\
            \hline
            Struktur der Vermittlung & Anwendung -> Rechnerinterne Darstellung \\
            \hline
            Ableitung von Unterthemen für einzelne Stunden & \\
            \hline
        \end{tabular}
    \end{center}
\end{block}

\begin{block}{Planungszeiträume}
    Baum?
    \begin{itemize}
        \item (Halb-)Jahresplanung
        \begin{itemize}
            \item Auf Schuljahr angepasst
            \item Ausgangspunkt der SuS
            \item Rahmenbedingungne durch das Fach
            \item Einplanen von zeitlichen Reserven
        \end{itemize}
        \item Unterrichtsreihen
        \begin{itemize}
            \item inhaltlich zusammenhängende Sequenz
            \item ein paar Stunden
        \end{itemize}
        \item Stundenentwurf
        \begin{itemize}
            \item Vorgehen in der einzelnen Stunde
            \item Sehr kleinschrittig
            \item Lernziele, Methoden, Sozialformen, Medien
        \end{itemize}
    \end{itemize}
\end{block}

% TODO: Unterrichtsreihen, Verlaufsformen?
Unterrichtsstufen können sich an Vorgehensmodellen orientieren:
\begin{block}{Modell von Roth}
    \centering
    \begin{tabular}{|p{10cm}|}
            \hline
            1. Stufe der Motivation \\
            \hline
            2. Stufe der Schwierigkeit \\
            \hline
            3. Stufe der Lösung \\
            \hline
            4. Stufe des Tuns und Ausführen \\
            \hline
            5. Stufe des Behaltens und Einübens \\
            \hline
            6. Stufe der Übertragung und Integration \\
            \hline
        \end{tabular}
\end{block}

\begin{block}{Wasserfallmodell}
    \centering
    \begin{tabular}{|p{10cm}|}
            \hline
            Problem \\
            \hline
            Analyse \\
            \hline
            Entwurf \\
            \hline
            Implementierung \\
            \hline
            Wartung \\
            \hline
        \end{tabular}
\end{block}

\begin{block}{Stoffauswahl}
    Stoffauswahl sollte sich orientieren an:
    \begin{itemize}
        \item Lernziele
        \item Erkenntnisse der Wissenschaft?
        \item Systematik, Planmäßigkeit
        \item Fasslichkeit
        \item Selbsttätigkeit der SuS
        \item Bezug zu anderen Fächern
        \item Comenius:
        \begin{itemize}
            \item vom Bekannten zum Neuen
            \item vom Nahen zum Fernen
            \item vom Einfachen zum Schwierigen
            \item vom Konkreten zum Abstrakten
        \end{itemize}
    \end{itemize}
    SuS sollen das Elementare selber 'ausgraben'
\end{block}

\section{Elementarisierung und didaktische Reduktion}
\begin{block}{Elementarisierung}
    \begin{itemize}
        \item Fördern vom 'ausgraben'
        \item Erschließung der Umwelt (Grundbedürfnis der SuS)
        \item Lehrkraft bereitet Stoff auf, damit selber entweckt werden kann
    \end{itemize}
\end{block}

\begin{block}{Didaktische Reduktion}
    \begin{itemize}
        \item Reduktion auf das Wesentliche, um Verständnis zu erleichtern
        \item Fachliche Richtigkeit, Ausbaufähigkeit und Angemessenheit muss gewahrt bleiben
        \item Beispiel: Algorithmus ohne Determinismusbegriff
    \end{itemize}
\end{block}
Erweiterung davon: Didaktische Rekonstruktion
\begin{itemize}
    \item Nicht nur Verständlich-Machen
    \item Inhalte werden bedeutsam und anschlussfähig gemacht
    \item Lernen wird ganzheitlich betrachtet
\end{itemize}

\section{Lerntheorien}
\begin{block}{Behaviorismus}
    \begin{itemize}
        \item Versuch, Psychologie Nachweisbarer zu Machen
        \item Beispiel: Pawlow/Watson
        \item Probleme:
        \begin{itemize}
            \item Nur Beobachtbares wird behandelt
            \item Subjektbezug fehlt
            \item Lernen wird als Reiz-Reaktions-Schema gesehen
        \end{itemize}
        \item Programmierte Unterweisung
        \item Ähnliches Verhältnis wie beim Programmieren
    \end{itemize}
\end{block}

\begin{block}{Kognitivismus}
    \begin{itemize}
        \item Blick ins Innere des Menschen
        \item Keine direkte Beobachtung möglich
        \item Gehirn will Überforderung reduzieren bzw. verhindern
        \item Also passiert Äquilibration (Streben nach Gleichgewicht)
        \item Durch Assimilation:
        \begin{itemize}
            \item Anpassen des aktuellen Modells an neue Informationen
            \item Beispiel: Zuerst sind alle Vierbeiner Hunde
            \item Dann wird Katze als Vierbeiner erkannt
            \item $\Rightarrow$ Modell wird angepasst
        \end{itemize}
        \item Durch Akkomodation:
        \begin{itemize}
            \item Einordnen neuer Informationen in bestehende Modelle
            \item Beispiel: Wenn eine Kuh so bezeichnet wird, werden die Unterschiede zum Hund direkt ins Modell eingebettet
        \end{itemize}
        \item Entwicklungsstadien erreichen erst ab 7 Jahren ein Stadium, bei dem Informatikunterricht sinnvoll ist
    \end{itemize}
\end{block}

\begin{block}{Bedeutungsvolles und Rezeptives Lernen}
    \begin{itemize}
        \item Wichtigstes beim Lernen ist die Verbindung zum alten Wissen
        \item $\Rightarrow$ Lehren ist das finden von 'Ankern' aus dem alten Wissen, an denen neues Wissen angedockt werden kann
        \item Advance organizer: Strukturierte Hilfen helfen den Lernenden, das neue Wissen einzuordnen
    \end{itemize}
    In der Informatik:
    \begin{itemize}
        \item konkretes Programmablaufmodell erleichtert Verstehen von Programmierbefehlen
    \end{itemize}
\end{block}

\begin{block}{Entdeckendes Lernen}
    \begin{itemize}
        \item Lerngegenstände können Lernenden in jeder Entwicklungsstufe gelehrt werden
        \item Lernen ist am Effektivsten, wenn Lernende selbstständig entdecken
    \end{itemize}
\end{block}

\begin{block}{Erkenntnisse aus kognitivistischer Perspektive}
    \begin{itemize}
        \item Lehrstoff in Zusammenhänge setzen
        \item Aneignung erleichtern durch Strukturierung
        \item Anknüpfung an Vorwissen
    \end{itemize}
\end{block}

\begin{block}{Konstruktivismus}
    \begin{itemize}
        \item Aktive Beteiligung der Lernenden
        \item Handlungsorientierung
        \item Möglichst viel selber erschließen lassen
        \item Lehrkraft: Organisator und Berater
        \item Wirklichkeitsnah
        \item Verschiedene Perspektiven zum selben Stoff
    \end{itemize}
\end{block}
\end{document}