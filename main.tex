\documentclass{article}

% config.tex
\usepackage{tcolorbox}
\usepackage{amssymb}
\usepackage{amsmath}
\usepackage[ngerman]{babel}
\usepackage[T1]{fontenc}
\usepackage[utf8]{inputenc}
\tcbuselibrary{listings, skins, breakable}

\newcounter{block}[section]
\renewcommand{\theblock}{\thesection.\arabic{block}}

% Automatisches Zurücksetzen von block bei jedem \section
\usepackage{etoolbox}
\preto\section{\setcounter{block}{0}}

\newtcolorbox{block}[1]{%
  enhanced,
  breakable,
  colback=green!5!white,
  colframe=green!65!black,
  coltitle=black,
  fonttitle=\bfseries,
  title=\thesection.\number\numexpr\value{block}+1\relax\ #1,
  before upper={},
  after upper={\stepcounter{block}},
  boxrule=0.8pt,
  arc=0mm,
  left=1mm,
  right=1mm,
  top=1mm,
  bottom=1mm,
}

\newenvironment{beweis}[1][Beweis]{
  \par\noindent\textbf{#1.}\\ 
}{
    \hfill \framebox[1.5ex][c]{}
    \par\vspace{2em}
    \noindent
}

\begin{document}

\begin{titlepage}
    \centering
    \includegraphics[width=0.3\textwidth]{Uni Logo.png} 
    \vspace*{2cm}
    
    {\Large\bfseries Einführung Fachdidaktik Informatik\par}
    \vspace{1.5cm}
    
    \textbf{Peter Minor}\\
    Sommersemester 2025
    
    \vfill    
    \vspace{1.5cm}
    {\large \today\par}
\end{titlepage}

\tableofcontents

\newpage

\section{Didaktische Dimensionen}
Die \textbf{Didaktik} fragt: \emph{Was}, \emph{wie}, \emph{warum}, \emph{wann}, \emph{wo}, \emph{mit wem}, \emph{womit} und \emph{für wen} gelehrt werden soll.

\begin{block}{Definition: Didaktische Dimensionen}
    Didaktische Dimensionen sind grundlegende Perspektiven, unter denen Unterricht geplant und reflektiert wird:
    \begin{itemize}
        \item \textbf{Inhalte}, \textbf{Ziele}, \textbf{Themen}
        \item \textbf{Methodik}, \textbf{Medien}, \textbf{Organisation des Lernens}
        \item \textbf{Bildung} als übergeordnetes Ziel
    \end{itemize}
\end{block}

\begin{block}{Didaktik vs Methodik}
    \textbf{Didaktik} beschäftigt sich mit der Frage \emph{was}, \emph{warum}, \emph{für wen} und \emph{mit welchem Ziel} gelehrt werden soll. \\
    \textbf{Methodik} hingegen fragt \emph{wie}, \emph{mit welchen Mitteln} und \emph{in welcher Form} der Unterricht konkret umgesetzt wird. \\[0.5em]
    \textit{Kurz:} Didaktik = \emph{Inhalts- und Zielperspektive},\\ \quad Methodik = \emph{Umsetzungs- und Prozessperspektive}.
\end{block}


\section{Hintergrund und Modelle}
\begin{block}{Didactica Magna – Comenius (1657)}
    Ziel: ``Die vollständige Kunst, alle Menschen alles zu lehren'' \\
    Leitideen:
    \begin{itemize}
        \item Rasch, angenehm und gründlich lehren
        \item Wahrheiten mit Beispielen aus mechanischen Künsten
        \item Feste Reihenfolge nach Alter, Zeit, Entwicklung
    \end{itemize}
\end{block}

\begin{block}{Allgemeine Didaktik}
    Basierend auf Lerntheorien. Modelle u.a.:
    \begin{itemize}
        \item Bildungstheoretisch / kritisch-konstruktiv
        \item Lerntheoretisch (behavioristisch, kognitivistisch, konstruktivistisch)
        \item Informationstheoretisch-kybernetisch
    \end{itemize}
    Konzepte:
    \begin{itemize}
        \item Kontextorientierung
        \item Forschend-entwickelnder Unterricht
        \item Projektorientierung
    \end{itemize}
    Prinzipien:
    \begin{itemize}
        \item An Grundideen orientieren
        \item Beziehungen herstellen
        \item Adäquat visualisieren
    \end{itemize}
\end{block}

\section{Informatikdidaktik}
\begin{block}{Fachdidaktik ist keine Abbilddidaktik}
    Ziel ist nicht die reine Weitergabe der Fachwissenschaft, sondern:
    \begin{itemize}
        \item Entwicklung von Welt- und Selbstverständnis Jugendlicher fördern
        \item Kooperative Reflexion mit Allgemeiner Didaktik und Bildungstheorie
        \item Fachinhalte auf Lebenswelt und Bildungsziele beziehen
    \end{itemize}
\end{block}

\begin{block}{Bezugswissenschaften}
    \begin{itemize}
        \item Fachwissenschaft Informatik
        \item Psychologie, Soziologie
        \item Allgemeine Didaktik, Bildungstheorie
    \end{itemize}
\end{block}

\section{Wissenschaftliche Perspektive}
\begin{block}{Forschungsdisziplin Didaktik der Informatik}
    \begin{itemize}
        \item Inhaltliche, methodische, mediale Konzepte
        \item Ziel: Qualitätssicherung informatischer Bildung
        \item Veranstaltungen: INFOS, DeLFI, WiPSCE
    \end{itemize}
\end{block}

\section{Normenproblem \& Bildungsbegriff}
\begin{block}{Normenproblem}
    Bildung ist wertgebunden und abhängig von gesellschaftlichen Idealen.\\
    Folge: Didaktische Forschung ist komplex und pluralistisch.
\end{block}

\begin{block}{Bildung}
    Bildung = eigenständige, individuelle Repräsentation von Kultur \\
    Begriffe wie ``Bildung'' oder ``Didaktik'' schwer ins Englische übersetzbar, da sie geisteswissenschaftlich tief verwurzelt sind.
\end{block}

\section{Gestaltung von Unterricht}
\begin{block}{Methoden in Vorlesungen (nach Weickert)}
    \textbf{Starter:} Kennenlernspiele, lebendige Statistik \\
    \textbf{Begleiter:} Mitdenken anregen, Brainstorming, Lernstopp \\
    \textbf{Evaluierer:} Fragebogen, Blitzlicht, ``Heute habe ich gelernt, ...''
\end{block}

\section{Lernumgebungen (Beispiele)}
\begin{itemize}
    \item \textbf{BlueJ}: Objektorientiertes Programmieren visuell erleben
    \item \textbf{Kara}: Steuerung eines Marienkäfers über Programmierung
    \item \textbf{PuMa}: Puppenhaus-Automation als niederschwellige Einführung in Programmierlogik
\end{itemize}

\section{Informatische Bildung}
\begin{block}{Allgemeinbildung und Schule}
    \begin{itemize}
        \item \textbf{Bildungsauftrag der Schule} nach Fend (1980): Qualifikation, Selektion, Sozialisation, Legitimation
        \item Allgemeinbildung als Vorbereitung auf Beruf, Studium und mündige Teilhabe an Gesellschaft
        \item Comenius: „Nur der gebildete Mensch ist Mensch“
        \item Klafki: \textbf{Epochaltypische Schlüsselprobleme} als Maßstab für relevante Bildungsinhalte
    \end{itemize}
\end{block}

\begin{block}{Beitrag der Informatik zur Allgemeinbildung}
    \begin{itemize}
        \item \textbf{IU als einziges Fach} mit technisch-naturwissenschaftlichem Fokus (Modrow 2005)
        \item Vermittlung von \textbf{Kernkonzepten der Informatik} ist zentral (Hartmann/Nievergelt)
        \item IU leistet Beitrag zur digitalen Mündigkeit und zur Reflexion gesellschaftlicher Entwicklungen
    \end{itemize}
\end{block}

\begin{block}{Problem der Realisierung}
    \begin{itemize}
        \item Zieltrias (Hartmann): Alltagsrelevanz, wissenschaftliches Verständnis, gesellschaftliche Reflexion
        \item Informatik gelingt diese Verknüpfung bislang unzureichend
    \end{itemize}
\end{block}

\section{Orientierungen und Konzepte}
\begin{block}{GI-Gesamtkonzept Informatische Bildung (2000)}
    \begin{itemize}
        \item Bildung durch Erschließen von Grundlagen, Methoden, Anwendungen und gesellschaftlicher Bedeutung von IS
        \item \textbf{Bewusstes Thematisieren} von Informatik erforderlich — keine bloße Techniknutzung
        \item \textbf{Abgrenzung} zur ITG (bedienorientiert) und Medienpädagogik
    \end{itemize}
\end{block}

\begin{block}{Vier Leitlinien (GI 2000)}
    \begin{enumerate}
        \item Interaktion mit Informatiksystemen
        \item Wirkprinzipien von IS
        \item Informatische Modellierung
        \item Wechselwirkungen: IS, Individuum, Gesellschaft
    \end{enumerate}
\end{block}

\section{Bildungsstandards und Schulstufen}

\begin{block}{Schwerpunkte je Schulstufe}
    \begin{itemize}
        \item \textbf{Primarstufe:} Werkzeuge, Grundkenntnisse, digitale Spaltung vermeiden
        \item \textbf{Sek I:} Handlungskompetenz, Systematisierung von Alltagserfahrungen
        \item \textbf{Sek II:} formale Methoden, informatisches Modellieren
    \end{itemize}
\end{block}

\begin{block}{Bildungsstandards (GI 2008 und KMK 2004)}
    \begin{itemize}
        \item Ergebnisorientierung (Kompetenzmodell nach Weinert)
        \item Drei Niveaustufen: Mindest-, Regel-, Maximalstandards
        \item Intention: \textbf{alle SuS} sollen IT zum Nutzen bewältigen können
    \end{itemize}
\end{block}

\section{Beruf und Wissenschaftspropädeutik}

\begin{block}{IU und Berufswelt}
    \begin{itemize}
        \item Förderung kreativen, algorithmischen Denkens, Transferfähigkeit, Teamarbeit
        \item Berufliche Orientierung durch technische Erfahrungen in der Schule
    \end{itemize}
\end{block}

\begin{block}{Wissenschaftspropädeutik}
    \begin{itemize}
        \item Aneignung von Grundlagenwissen, Reflexionsfähigkeit, Lernstrategien
        \item Informatik als Zugang zu ingenieurwissenschaftlichem Denken
    \end{itemize}
\end{block}

\section{Internationale Perspektiven}
\begin{block}{UNESCO ICT Curriculum (2000)}
    \begin{enumerate}
        \item ICT Literacy (Computer bedienen)
        \item ICT in Fächern anwenden
        \item ICT fachübergreifend integrieren
        \item ICT-Spezialisierung
    \end{enumerate}
\end{block}

\begin{block}{Being Fluent with IT (NRC 1999)}
    \begin{itemize}
        \item Literacy (Fakten), Capabilities (Fähigkeiten), Concepts (Konzepte)
        \item Ziel: dauerhafte, tiefgreifende IT-Kompetenz, nicht nur Bedienung
    \end{itemize}
\end{block}

\section{Digitale Mündigkeit und aktuelle Entwicklungen}

\begin{block}{Digitale Mündigkeit}
    \begin{itemize}
        \item Kritische Reflexion, Urteilskompetenz, gesellschaftliche Verantwortung
        \item Kompetenzrahmen: Problemlösen, Automatisierung, Algorithmisches Denken
    \end{itemize}
\end{block}

\begin{block}{Rahmen und Strategien (Auswahl)}
    \begin{itemize}
        \item Medienkompetenzrahmen NRW (2018–)
        \item KMK-Strategie „Bildung in der digitalen Welt“ (2016–)
        \item EU DigComp 2.2 (2022), UNESCO Framework (2018)
    \end{itemize}
\end{block}

\section{Thema C: Was ist Informatik}
\begin{block}{Definition und Abgrenzung}
    \begin{itemize}
        \item \textbf{Informatik} beschäftigt sich mit der Darstellung, Speicherung, Übertragung und Verarbeitung von Information.
        \item Die Fragestellungen und Inhalte der Fachwissenschaft Informatik unterscheiden sich von populären Vorstellungen (z.B. Office-Anwendungen, reine Mediennutzung oder Elektrotechnik zählen nicht zur Informatik).
        \item Informatik ist sowohl eine \textbf{Grundlagenwissenschaft} als auch eine \textbf{Ingenieurwissenschaft}.
        \item Informatik betrachtet Information aus verschiedenen Perspektiven: technisch, personal, organisationsbezogen und medial.
    \end{itemize}
\end{block}

\begin{block}{Was gehört (nicht) zur Informatik?}
    \centering
    \begin{tabular}{|p{5cm}|p{5cm}|}
        \hline
        \textbf{Das ist Informatik} & \textbf{Das ist keine Informatik} \\
        \hline
        Algorithmisches Denken & Office-Handhabung \\
        Programmieren & Elektrotechnik \\
        Hardware und Software & Digitale Medienkunde \\
        Theoretische Informatik & Internetanwendungen \\
        Datenmanagement & Wie baue ich einen PC \\
        Netzwerke & Homepage-Design \\
        Informationsverarbeitung & Toaster \\
        Datensicherheit & \\
        Informatik und Gesellschaft & \\
        \hline
    \end{tabular}
\end{block}

\begin{block}{Historische Entwicklung}
    \begin{itemize}
        \item \textbf{Charles Babbage}: Difference Engine und Analytical Engine als erste Konzepte universeller Maschinen.
        \item \textbf{Konrad Zuse}: Erste programmierbare Rechner (Z1, Z3), Entwicklung des Plankalküls als früher Programmiersprache.
        \item Entwicklung von mechanischen und elektromechanischen Rechnern (z.B. MARK I, ENIAC) zur von-Neumann-Architektur und modernen Computern.
        \item Entdeckung des Transistors und Miniaturisierung ermöglichen Mikroprozessoren und heutige Computertechnik.
        \item Vernetzung von Rechnern (z.B. ARPANET, später Internet) und Entwicklung von Software prägen die Informatik maßgeblich.
    \end{itemize}
\end{block}

\begin{block}{Theoretische Grundlagen}
    \begin{itemize}
        \item \textbf{Formale Logik} bildet die Grundlage der Informatik (von Aristoteles bis zur modernen Logik).
        \item \textbf{Kalkül} (Leibniz) und \textbf{Algorithmus} (Turing) als zentrale Konzepte:
        \begin{itemize}
            \item Allgemeinheit, Endlichkeit, Determiniertheit, Terminierung, Determinismus
        \end{itemize}
        \item Der \textbf{Gödelsche Unvollständigkeitssatz} zeigt die Grenzen formaler Systeme auf.
        \item Die \textbf{Turingmaschine} dient als Modell für Berechenbarkeit und Algorithmik.
    \end{itemize}
\end{block}

\begin{block}{Informatik und Gesellschaft}
    \begin{itemize}
        \item Informatik prägt Berufswelt, Kommunikation und gesellschaftliche Strukturen grundlegend.
        \item Digitale Mündigkeit und kritische Reflexion sind wichtige Ziele informatischer Bildung.
        \item Informatik ist interdisziplinär mit Psychologie, Soziologie und Didaktik verbunden.
        \item Das sogenannte \textbf{Normenproblem}: Bildung ist wertgebunden und von gesellschaftlichen Idealen geprägt, was die didaktische Forschung komplex und pluralistisch macht.
    \end{itemize}
\end{block}

\section{Elemente der Unterrichtsgestaltung}
\begin{block}{Unterricht}
    \begin{itemize}
        \item Gezielte, geplante Vermittlung von Wissen, Fähigkeiten und praktischem können
        \item Keine zufälligen Belehrungen oder Hinweise
        \item Kennzeichen von Unterricht an Schulen:
        \begin{itemize}
            \item Pädagogische Gerichtetheit
            \item Planmäßigkeit
            \item Institutionalisierung
            \item Verberuflichung
        \end{itemize}
    \end{itemize}
\end{block}

Unterricht ohne Ziel: Diffus

\begin{block}{Lernziele}
    Lernzieldimensionen:
    \begin{itemize}
        \item Kognitive Lernziele
        \begin{itemize}
            \item Wissen, Verstehen, Anwenden, Analysieren, Synthesieren, Bewerten
        \end{itemize}
        \item Affektive Lernziele
        \begin{itemize}
            \item Beobachten, Beantworten, Bewerten, ..., Weltanschauung
        \end{itemize}
        \item Psychomotorische Lernziele
        \begin{itemize}
            \item Imitatieren, Manipulieren, Präzisieren, ..., Verinnerlichung
        \end{itemize}
    \end{itemize}
\end{block}

\begin{block}{AFBs}
    \textbf{Allgemeine Fachliche Begriffe} (AFBs) sind zentrale Begriffe der Informatik, die in der Schule vermittelt werden sollen. Sie dienen als Grundlage für die Entwicklung von Kompetenzen und Fähigkeiten im Umgang mit informatischen Systemen.
    \begin{itemize}
        \item AFB I: Wissen wiedergeben, Methoden anwenden (30-40\%)
        \item AFB II: Probleme lösen, Konzepte verstehen (50-60\%)
        \item AFB III: Systeme analysieren, kritisch reflektieren (10-20\%)
    \end{itemize}
\end{block}

\begin{block}{Kompetenzmodell}
    \begin{itemize}
        \item Fähigkeit, persönliches, berufliches und gesellschaftliches Leben zu führen
        \item Aufgeteilt in:
        \begin{itemize}
            \item Sachkompetenz: Kentnisse und Einsichten
            \item Sozialkompetenz: Fähigkeit, eigene Ziele im Einklang mti anderen Beteiligten zu verfolgen
            \item Methodenkompetenz: Fähigkeit, eigenen Lernprozess zu gestalten
            \item Personale Kompetenz: Einstellungen, Motivationen, die das Handeln beeinflussen(Selbstvertrauen)
        \end{itemize}
    \end{itemize}
\end{block}

\begin{block}{Bildungsstandards}
    Kamen durch Pisa 2000, KMK wollte einheitl. Standards
    \begin{itemize}
        \item Kompetenzorientierung anstatt INPUT-Orientierung
        \item Mindeststandards für alle Schüler
        \item Regelstandards für die meisten Schüler
        \item Maximalstandards für die leistungsstärksten Schüler
    \end{itemize}
    Kompetenzorientierung theoretisch zwar gut, aber meist bleiben die Inhalte die selben
\end{block}

\begin{block}{Gegenstand - Inhalt - Thema}
    Thema benennt einen Inhalt, der an einem Gegenstand vermittelt wird.
    Beispiele für Inhaltsbereiche aus der Informatik:
    \begin{itemize}
        \item DB und Informationssysteme
        \item Rechnerarchitektur
        \item Geschichte der Informatik
        \item Sprach- und Signalverarbeitung
    \end{itemize}
    Beispielprozess für Themenfindung:
    \begin{center}
        \begin{tabular}{|p{5cm}|p{5cm}|}
            \hline
            \textbf{Prozess zur Themenfindung} & \textbf{Beispiel} \\
            \hline
            Idee: Thema wird grob formuliert & Verarbeitung von Bildern mit Informatiksystemen \\
            \hline
            Was soll mit dem Thema vermittelt werden? & \\
            \hline
            Lernziele & Erfahren, dass die Verarbeitung von grafischen Daten zur Veränderung der Informationen führen kann. \\
            \hline
            Lerninhalte & Flussdiagramm zur Notation von Algorithmen \\
            \hline
            Fächerverbindungen & Computerkunst, Optik \\
            \hline
            Wie können die Lerninhalte interessant vermittelt werden? & \\
            \hline
            Alltagsbezüge & Computergrafik \\
            \hline
            Medien & Paint \\
            \hline
            Struktur der Vermittlung & Anwendung -> Rechnerinterne Darstellung \\
            \hline
            Ableitung von Unterthemen für einzelne Stunden & \\
            \hline
        \end{tabular}
    \end{center}
\end{block}

\begin{block}{Planungszeiträume}
    Baum?
    \begin{itemize}
        \item (Halb-)Jahresplanung
        \begin{itemize}
            \item Auf Schuljahr angepasst
            \item Ausgangspunkt der SuS
            \item Rahmenbedingungne durch das Fach
            \item Einplanen von zeitlichen Reserven
        \end{itemize}
        \item Unterrichtsreihen
        \begin{itemize}
            \item inhaltlich zusammenhängende Sequenz
            \item ein paar Stunden
        \end{itemize}
        \item Stundenentwurf
        \begin{itemize}
            \item Vorgehen in der einzelnen Stunde
            \item Sehr kleinschrittig
            \item Lernziele, Methoden, Sozialformen, Medien
        \end{itemize}
    \end{itemize}
\end{block}

% TODO: Unterrichtsreihen, Verlaufsformen?
Unterrichtsstufen können sich an Vorgehensmodellen orientieren:
\begin{block}{Modell von Roth}
    \centering
    \begin{tabular}{|p{10cm}|}
            \hline
            1. Stufe der Motivation \\
            \hline
            2. Stufe der Schwierigkeit \\
            \hline
            3. Stufe der Lösung \\
            \hline
            4. Stufe des Tuns und Ausführen \\
            \hline
            5. Stufe des Behaltens und Einübens \\
            \hline
            6. Stufe der Übertragung und Integration \\
            \hline
        \end{tabular}
\end{block}

\begin{block}{Wasserfallmodell}
    \centering
    \begin{tabular}{|p{10cm}|}
            \hline
            Problem \\
            \hline
            Analyse \\
            \hline
            Entwurf \\
            \hline
            Implementierung \\
            \hline
            Wartung \\
            \hline
        \end{tabular}
\end{block}

\begin{block}{Stoffauswahl}
    Stoffauswahl sollte sich orientieren an:
    \begin{itemize}
        \item Lernziele
        \item Erkenntnisse der Wissenschaft?
        \item Systematik, Planmäßigkeit
        \item Fasslichkeit
        \item Selbsttätigkeit der SuS
        \item Bezug zu anderen Fächern
        \item Comenius:
        \begin{itemize}
            \item vom Bekannten zum Neuen
            \item vom Nahen zum Fernen
            \item vom Einfachen zum Schwierigen
            \item vom Konkreten zum Abstrakten
        \end{itemize}
    \end{itemize}
    SuS sollen das Elementare selber 'ausgraben'
\end{block}

\section{Elementarisierung und didaktische Reduktion}
\begin{block}{Elementarisierung}
    \begin{itemize}
        \item Fördern vom 'ausgraben'
        \item Erschließung der Umwelt (Grundbedürfnis der SuS)
        \item Lehrkraft bereitet Stoff auf, damit selber entweckt werden kann
    \end{itemize}
\end{block}

\begin{block}{Didaktische Reduktion}
    \begin{itemize}
        \item Reduktion auf das Wesentliche, um Verständnis zu erleichtern
        \item Fachliche Richtigkeit, Ausbaufähigkeit und Angemessenheit muss gewahrt bleiben
        \item Beispiel: Algorithmus ohne Determinismusbegriff
    \end{itemize}
\end{block}
Erweiterung davon: Didaktische Rekonstruktion
\begin{itemize}
    \item Nicht nur Verständlich-Machen
    \item Inhalte werden bedeutsam und anschlussfähig gemacht
    \item Lernen wird ganzheitlich betrachtet
\end{itemize}

\section{Lerntheorien}
\begin{block}{Behaviorismus}
    \begin{itemize}
        \item Versuch, Psychologie Nachweisbarer zu Machen
        \item Beispiel: Pawlow/Watson
        \item Probleme:
        \begin{itemize}
            \item Nur Beobachtbares wird behandelt
            \item Subjektbezug fehlt
            \item Lernen wird als Reiz-Reaktions-Schema gesehen
        \end{itemize}
        \item Programmierte Unterweisung
        \item Ähnliches Verhältnis wie beim Programmieren
    \end{itemize}
\end{block}

\begin{block}{Kognitivismus}
    \begin{itemize}
        \item Blick ins Innere des Menschen
        \item Keine direkte Beobachtung möglich
        \item Gehirn will Überforderung reduzieren bzw. verhindern
        \item Also passiert Äquilibration (Streben nach Gleichgewicht)
        \item Durch Assimilation:
        \begin{itemize}
            \item Anpassen des aktuellen Modells an neue Informationen
            \item Beispiel: Zuerst sind alle Vierbeiner Hunde
            \item Dann wird Katze als Vierbeiner erkannt
            \item $\Rightarrow$ Modell wird angepasst
        \end{itemize}
        \item Durch Akkomodation:
        \begin{itemize}
            \item Einordnen neuer Informationen in bestehende Modelle
            \item Beispiel: Wenn eine Kuh so bezeichnet wird, werden die Unterschiede zum Hund direkt ins Modell eingebettet
        \end{itemize}
        \item Entwicklungsstadien erreichen erst ab 7 Jahren ein Stadium, bei dem Informatikunterricht sinnvoll ist
    \end{itemize}
\end{block}

\begin{block}{Bedeutungsvolles und Rezeptives Lernen}
    \begin{itemize}
        \item Wichtigstes beim Lernen ist die Verbindung zum alten Wissen
        \item $\Rightarrow$ Lehren ist das finden von 'Ankern' aus dem alten Wissen, an denen neues Wissen angedockt werden kann
        \item Advance organizer: Strukturierte Hilfen helfen den Lernenden, das neue Wissen einzuordnen
    \end{itemize}
    In der Informatik:
    \begin{itemize}
        \item konkretes Programmablaufmodell erleichtert Verstehen von Programmierbefehlen
    \end{itemize}
\end{block}

\begin{block}{Entdeckendes Lernen}
    \begin{itemize}
        \item Lerngegenstände können Lernenden in jeder Entwicklungsstufe gelehrt werden
        \item Lernen ist am Effektivsten, wenn Lernende selbstständig entdecken
    \end{itemize}
\end{block}

\begin{block}{Erkenntnisse aus kognitivistischer Perspektive}
    \begin{itemize}
        \item Lehrstoff in Zusammenhänge setzen
        \item Aneignung erleichtern durch Strukturierung
        \item Anknüpfung an Vorwissen
    \end{itemize}
\end{block}

\begin{block}{Konstruktivismus}
    \begin{itemize}
        \item Aktive Beteiligung der Lernenden
        \item Handlungsorientierung
        \item Möglichst viel selber erschließen lassen
        \item Lehrkraft: Organisator und Berater
        \item Wirklichkeitsnah
        \item Verschiedene Perspektiven zum selben Stoff
    \end{itemize}
\end{block}

\begin{block}{Interaktionistischer Konstruktivismus}
    \begin{itemize}
        \item Lernen passiert in Interaktion mit der Welt
        \begin{itemize}
            \item Entdecken der Welt(Rekonstruieren)
            \item Erfinden der Welt(Konstruieren)
            \item Kritisieren der Welt(Dekonstruieren)
        \end{itemize}
        \item Welt ist hier Kultur, Soziales etc.
        \item Methodenpool?
    \end{itemize}
\end{block}

\section{Unterichtsmethoden, -prinzipien und -konzepte/Modelle}

\begin{block}{Unterrichtsmethoden}
    \begin{itemize}
        \item Formen und Verfahren, mit denen SuS und Lehrkräfte gemeinsam Lernen
        \item Das geschieht durch die auseinandersetzung mit der natürlichen und gesellschaftl. Realität
    \end{itemize}
\end{block}

\begin{block}{unterrichtsmethodische Reflexion}
    \begin{itemize}
        \item Handlungssituationen
        \begin{itemize}
            \item Zeitlich begrenzte Interaktionseinheiten, die bewusst gestaltet und mit Sinn und Bedeutung belegt sind
            \item Beispiel: Frage stellen und antworden, Arbeitsauftrag formulieren, Schummeln
            \item Informatik: Interaktionen mit Computer, UML-Diagramm zeichnen
        \end{itemize}
        \item Arbeitsformen/Handlungsmuster
        \begin{itemize}
            \item Historische Formen der Wirklichkeitsaneignung
            \item Haben Anfang und Ende, sind Zielgerichtet
            \item Beispiel: Unterrichtsgespräch, Diskussion, Texterarbeitung
            \item Informatik: Projektorientiertes Arbeiten, Programmieren
            \item Vielfalt der Methoden ist wichtig
        \end{itemize}
        \item Unterrichtsschritte
        \begin{itemize}
            \item Siehe 18.7: Planungszeiträume
        \end{itemize}
        \item Sozialformen
        \begin{itemize}
            \item Genau 4!
            \item Frontalunterricht
            \item Partnerarbeit
            \item Gruppenunterricht
            \item Einzelarbeit
            \item Informatik: Partner- bzw. Gruppenarbeit bevorzugt
            \item Hochmotivierte SuS können Einzelarbeit bevorzugen
        \end{itemize}
        \item Methodische Großformen
        \begin{itemize}
            \item Feste, Bewährte Konzepte, für die Strukturierung größerer Lernvorhaben
            \item Beispiel: Projektarbeit, Stationenlernen
        \end{itemize}
    \end{itemize}
\end{block}

\begin{block}{Lernaufgaben}
    \begin{itemize}
        \item Aufgabe, deren Lösung neues Wissen bzw. Können benötigt
        \item Lernerfolg ergibt sich aus der (korrekten und vollst.) Lösung der Aufgabe
        \item Erfolg der Bearbeitung kann vom Lernenden selbst erkannt werden
        \item Hat Bezug zu beruflichen Aufgaben bzw. Tätigkeiten
    \end{itemize}
\end{block}

\begin{block}{Gruppenarbeit}
    \begin{itemize}
        \item Aufgaben dürfen nicht alleine lösbar sein
        \item mindestens teilweise gemeinsames Arbeiten notw.
        \item Gruppen sollen zusammengesetzt sein, sodass
        \begin{itemize}
            \item unterschiedl. Vorraussetzungen und Kenntnisse von den Mitgliedern erfüllt werden
            \item Treffen außerhalb des Unterrichts möglich ist
            \item Es sollen keine 'Außenseiter' existieren
        \end{itemize}
        \item Beispiel: Gruppenpuzzle(=autonomes Lernen + Gruppenarbeit)
    \end{itemize}
\end{block}

\begin{block}{Unterrichts- und didaktische Prinzipien}
    \begin{itemize}
        \item Regeln für Gestaltung und beurteilung von Unterricht
        \item Beruht auf normativen Überlegungen und praktischen (Unterrichts-) Erfahrungen
        \item Beispiele für Unterrichtsprinzipien, die auf der Lernpsychologie basieren:
        \begin{itemize}
            \item Prinzip der Motivierung: Lernende sollen intrinsisch motiviert sein
            \item Prinzip der Veranschaulichung: Lerninhalte sollen konkret und greifbar sein
            \item Differenzierung: Unterricht soll auf individuelle Lernvoraussetzungen eingehen
            \item Prinzip der Aktivierung: Lernende sollen aktiv am Lernprozess beteiligt sein
        \end{itemize}
        \item Informatikdidaktische Prinzipien:
        \begin{itemize}
            \item SuS sollen selbst tätig werden(insb. konkretes Tun, Umformen, Ausprobieren)
            \item Lernstoff soll wiederholt, ggf. mit erhöhtem Niveau erweitert werden
            \item Abstrakte Inhalte sollen durch Bilder, Modelle usw. begreifbar gemacht werden.
            \item Lerninhalte soll an der Lebenswelt der SuS orientiert sein
            \item Lerninhalte sollen vernetzt sein(untereinander oder mit anderen Fächern)
            \item Lernprozesse sollen klare Ziele haben
            \item Lernprozesse sollen so strukturiert sein, dass sie der natürlichen Entwicklung der Inhalte folgen
        \end{itemize}
    \end{itemize}
\end{block}

\begin{block}{Fachsprache und Begriffsverständnis}
    Es gibt drei Ebenen von Sprache, die für (Informatik-)Lehrkräfte wichtig sind:
    \begin{itemize}
        \item Umgangssprache der SuS, z.B. 'Computer ist abgestürzt', 'der Kreis geht nicht mehr weg'
        \item Fachsprache(der Informatik), z.B. 'Algorithmus', 'Bedingung'
        \item Unterrichtssprache, hauptsächlich Verbindungsebene, durch Erklärungen, Analogien und Beispiele
    \end{itemize}
    Zum Heranführen an die Fachsprache:
    Stufenmodell zum Lernen von Begriffen:
    \begin{itemize}
        \item Intuitives Verständnis(aus der Umgangssprache)
        \item Inhaltliches Verständnis(der Begriff wird bewusst wahrgenommen und mit Beispielen verbunden)
        \item Integriertes Verständnis(Verbindung mit anderen Begriffen existiert)
        \item Strukturelles Verständnis(der Begriff wird benutzt, u.a. für Problemlösungen)
        \item Formales Verständnis(formale Definition, inklusive Beweise)
    \end{itemize}
\end{block}

\begin{block}{Abstraktion und Repräsentationsebenen}
    Abstraktion ist die Reduktion auf das wesentliche.
    In der Informatik:
    \begin{itemize}
        \item Reduktion aus der realen Welt in ein (informatisches) Modell
        \item Formalisierung
        \item Kapselung, Datenstrukturen
    \end{itemize}
    Repräsentationsebenen helfen bei der Abstraktion, indem sie schrittweise aufgbeaut wird.
    Die Repräsentationsebenen sind:
    \begin{itemize}
        \item Enaktiv: Lernen durch Handeln(sehr konkret in der Lebenswelt)
        \item Ikonisch: Lernen durch Bilder, Diagramme
        \item Symbolisch: Lernen durch abstrakte Sprache(z.B. Code)
    \end{itemize}
\end{block}

\begin{block}{Motivation und Aufmerksamkeit}
    Motivation ist die Bereitschaft, ein gewisses Verhalten zu zeigen
    \begin{itemize}
        \item Für Lehrkräfte bzw. Lernende: Lernmotivation und Leistungsmotivation
        \item Intrinsische Motivation: Interesse am Lernstoff selbst
        \item Extrinsische Motivation: Belohnung, Noten, Anerkennung
        \item Wechseln der Methodik hilft beim Aufrechterhalten der Motivation
    \end{itemize}
    Aufmerksamkeit ist die Fähigkeit, sich auf bestimmte Stimuli der Umwelz zu konzentrieren
    \begin{itemize}
        \item Abweichendes erhält mehr Aufmerksamkeit(Farben, Bewegungen, Geräusche)
        \item Aufmerksamkeit kann nur einem Inhalt gleichzeitig gelten
        \item Wechsel der Aufmerksamkeit ist anstrengend $\Rightarrow$ schneller Wechsel von Inhalten führt zur Ermüdung
    \end{itemize}
\end{block}

\begin{block}{Lesen vor Schreiben}
    Im Sprachunterricht wird immer zuerst Lesen und dann Schreiben gelernt.
    In der Informatik Sinnvoll?
    \begin{itemize}
        \item Für das Lernen einer Programmiersprache sinnvoll
        \item Software(re)engineering sinnvoll, aber bisher wenig betrachtet
        \item Dekonstruktion von (fremd-)Software als Unterrichtsmethode
    \end{itemize}
\end{block}

\begin{block}{Freiraum für Kreativität in Informatikunterricht}
    \begin{itemize}
        \item Entwicklung von Hard- und Software ist inhärent kreativ
        \item $\Rightarrow$ möglichst viel Freiraum in diesen Aufgaben lassen
        \item Optimierung von Verfahren, Modellen etc. ist auch kreativ
        \item Projektarbeit optimal für kreative Arbeit
    \end{itemize}
\end{block}

\begin{block}{Unterrichtskonzepte und -Modelle}
    Sind Orientierungen methodischen Handelns, beruht auf Unterrichtsprinzipien und anderen Theoretischen Grundlagen.
    In der Informatik:
    \begin{itemize}
        \item Entdeckendes Lernen
        \begin{itemize}
            \item SuS lösen selbstständig Probleme
            \item Lehrkraft als Lernimpuls oder -arragement
            \item kein festes Schema, Lernprozess wird nur begleitet
        \end{itemize}
        \item Projektorientiertes Lernen
        \begin{itemize}
            \item Projektwoche
            \item Projektunterricht
            \item Lernstoff wird an einem Projekt erarbeitet oder verfestigt
        \end{itemize}
        \item Handlungsorientiertes Lernen
        \begin{itemize}
            \item Erprobung eines Handlungsprozesses
            \item Anwendung der Kenntnisse und Verallgemeinerungen
            \item auch Teil von Projektunterricht, Stationenlernen, Lernen durch Lehren etc.
        \end{itemize}
        \item Erfahrungsbasiertes Lernen
        \begin{itemize}
            \item Lernzyklus
            \begin{itemize}
                \item Beginnt mit konkteten Erfahrungen der SuS
                \item SuS reflektieren diese Erfahrungen
                \item SuS abstrahieren aus den Erfahrungen
                \item SuS Experimentieren mit den abstrahierten Konzepten $\Rightarrow$ neue Erfahrungen
            \end{itemize}
            \item Die Lernstile der SuS müssen berücksichtigt werden
            \begin{itemize}
                \item Divergierer
                \item Assimilierer
                \item Konvergierer
                \item Akkomodierer
            \end{itemize}
        \end{itemize}
        \item Problemorientiertes Lernen
        \begin{itemize}
            \item Anspruchsvolles Lernen
            \item Allgemein: Analyse des Problems, Entwickeln einer Lösestrategie, Anwenden der Strategie
            \item Lernen findet hauptsächlich in der Entwicklungsphase statt $\Rightarrow$ Probleme dürfen nicht routinemäßig gelöst werden können
        \end{itemize}
        \item Genetisches Lernen(genetisches Prinzip)
        \begin{itemize}
            \item 'Nachspielen' der Entwicklung der Lerninhalte
            \item SuS sollen die Inhalte selbst entdecken
            \item Ziel: Nachvollzug der Entwicklung des Fachs(in der Informatik)
        \end{itemize}
    \end{itemize}
\end{block}

\begin{block}{Unterrichtsmedien}
    Medien können zu verschiedenen Zwecken eingesetzt werden:
    \begin{itemize}
        \item Informationsvermittlung
        \item Motivationshilfe(durch Aktivität, Anschauung, Ästhetik)
        \item Initiierung und Steuerung von Lernprozessen(u.A. für Differenzierung und Individualisierung sinnvoll)
        \item Akzentsetzung, Aufnahmeerleichterung
    \end{itemize}
    Typische Medien (im Informatikunterricht) sind:
    \begin{itemize}
        \item Wandtafel
        \begin{itemize}
            \item Erfordert aufwendige Planung von einem Tafelbild
            \item sollte korrekt, übersichtlich und verständlich sein
            \item 'Interaktive Tafel' als moderne Variante
            \begin{itemize}
                \item Erlaubt Einbindung von digitalen Medien
                \item Inhalte verschieben, kopieren, einfügen
                \item Interaktion mit Software an der Tafel
                \item SuS schreiben weniger
                \item $\Rightarrow$ sind sehr unterschiedlich von der 'herkömmlichen' Tafel und benötigen eigene Didaktik
            \end{itemize}
        \end{itemize}
        \item Schulbücher
        \begin{itemize}
            \item soll mit den Rahmenplänen übereinstimmen
            \item soll an die SuS angepasst sein
            \item soll gute methodische Gestaltung haben
            \item soll ein gutes Didaktisches Konzept haben
            \item gutes Hilfemittel für die Unterrichtsplanung
        \end{itemize}
        \item Tranzparente(Overheadprojektor)
        \begin{itemize}
            \item Übelst veraltet wtf?!
            \item Abdecktechnik, Markierungstechnik, Aufbautechnik, Ergänzungstechnik
        \end{itemize}
        \item Mechanische Unterrichtsmittel
        \begin{itemize}
            \item z.B. zum Veranschaulichen von Algorithmen
        \end{itemize}
        \item Software
        \item Hardware
        \item Programmiersprachen
        \begin{itemize}
            \item Die konktete Programmiersprache darf die Lernziele nicht überschatten
            \item Syntax, Datenstrukturen usw. sollten nicht im Vordergrund Synthesieren
            \item Aber das einzige Tool, was Algorithmen und Datenstrukturen gut darstellen und umsetzen kann
            \item $\Rightarrow$ Programmiersprache sollte nicht zu komplex sein
        \end{itemize}
    \end{itemize}
\end{block}

\begin{block}{Lernkontrollen}
    Kontrollieren den aktuellen Stand des Lernprozesses. Auf den Ergebnissen  beruht das weitere Vorgehen
    \begin{itemize}
        \item Leistung ist
        \begin{itemize}
            \item Ergebnis und Vollzug einer Tätigkeit, die mit Anstrengung verbunden ist
            \item Ergebnis ist messbar
            \item Abhängig vom Lernangebot
        \end{itemize}
        \item Beurteilung der Leistung orientiert sich an Normen, Durchschnitt oder Fortschritt
        \item Noten sind als Beurteilung einem Wortgutachten gegenübergestellt, beides mit Vor- und Nachteilen
        \item Formen von Lernkontrollen
        \begin{itemize}
            \item Schriftl. Arbeiten(Vor allem Sach- und Methodenkompetenz)
            \item Mündliche Beiträge(Personal- und Sozialkompetenz Zusätzl.), Abhängig vom Verhältnis zwischen Lehrkraft und SuS
            \item praktische Arbeiten(z.B. am Rechner)
            \item Portfolios/Leistungsmappen(s.u.)
            \item Lernkontrakte('Verträge' zwischen Lehrkraft und SuS)
            \item Selbst-, Peerbewertung
            \item Lerntagebuch
        \end{itemize}
        \item Bewertet werden können:
        \begin{itemize}
            \item Kenntnisse(Korrektheit, Umfang)
            \item selbstständiges Arbeiten
            \item Fachsprachkentnisse
        \end{itemize}
        \item Portfolios/Leistungsmappen
        \begin{itemize}
            \item Ausgewählte Arbeiten
            \item Auswahl durch SuS, aber Vorgaben durch Lehrkraft
            \item Umfangreiche Portfolios können aufgeteilt werden
            \begin{itemize}
                \item Vorzeigeportfolio
                \item Entwicklungsportfolio
                \item Prüfungsportfolio
            \end{itemize}
            \item Problem: fehlende Standards
        \end{itemize}
    \end{itemize}
\end{block}

\begin{block}{Leistung beurteilen}
    \begin{itemize}
        \item Objektive Leistungserschätzung Ziel, aber leider unmöglich
        \item Versuch, möglichst nah dran zu Kompetenzrahmen
        \item Oft wird eine Kombination aus Kriterien- und sozial orientierte Bezugsnorm herangezogen
        \item Klassendurschnitt als soziale Bezugsnorm, daran kann normalisiert werden
        \item Noten sind oft das Resultat einer Leistungsbeurteilung
        \begin{itemize}
            \item Teilen der Leistungen in Notenbereiche
            \begin{itemize}
                \item 1: Sehr gut, entspricht in besonderem Maße den Anforderungen
                \item 2: Gut, entspricht den Anforderungen voll
                \item 3: Befriedigend, entspricht im Allgemeinen den Anforderungen
                \item 4: Ausreichend, entspricht den Anforderungen in wesentlichen Teilen, kann aber Mängel aufweisen
                \item 5: Mangelhaft, entspricht nicht den Anforderungen, weist aber notwendige Grundkenntnisse auf, sodass die Mängel absebahr behoben werden können
                \item 6: Ungenügend, entspricht nicht den Anforderungen, selbst Grundkenntnisse sind Mangelhaft
            \end{itemize}
            \item Noten sind kein guter Indikator für Lernerfolg!
            \item Vergleichbarkeit ist nicht gegeben
            \item Alternative: Tranzparente Leistungsbeurteilung(u.a. durch Pisa bestätigt)
            \item Aktive Beteiligung der SuS an der Leistungskontrolle, Reflexion und Bewertung
        \end{itemize}
        \item Qualitätskriterien für Leistungsmessung
        \begin{itemize}
            \item Objektivität: Unabh. von der Person, die die Leistung misst
            \item Reliabilität: Bei den gleichen Leistung ist die Bewertung immer gleich
            \item Validität: Die Leistung misst auch das, was sie messen soll
        \end{itemize}
        \item In der Schule kaum zu realisieren, Subjektivität großer Faktor in der Beurteilung
        \item Es wird nur Performanz gemessen, daraufhin wird ein Rückschluss auf die Kompetenz gezogen
    \end{itemize}
\end{block}

\begin{block}{Zweck von Aufgaben}
    \begin{itemize}
        \item Erkundungsaufgaben
        \begin{itemize}
            \item leichter Zugang
            \item Offenheit, vor allem bzgl.
            \begin{itemize}
                \item Ausgangssituation
                \item Weg
                \item oder Ziel/Ergebnis
            \end{itemize}
            \item Es ex. eine Barriere, dessen Überwindung neue Kenntnisse braucht
            \item Lösung führt zwanghaft durch bedeutsamen Inhalt
            \item Sollte Authentisch sein, also Bezug zur Lebenswelt der SuS haben
            \item Authentizität ist ein Qualitätskriterium für Aufgaben
        \end{itemize}
        \item Sammlungsaufgaben
        \begin{itemize}
            \item SuS sammeln Informationen, sichten und Systematisieren diese
            \item Wie Erkundungsaufgabe, aber mehr Vorgabe, mit welchen Inhalten gearbeitet werden soll
        \end{itemize}
    \end{itemize}
\end{block}

\begin{block}{Offenheit von Aufgaben}
    \begin{itemize}
        \item Eisn der Qualitätskriterien von Aufgaben
        \item Mit Offenheit muss man als Lehrkraft umgehen können
        \item $\Rightarrow$ Flexibilität in der Lösungsdarstellung
        \begin{itemize}
            \item Offene Aufgaben
            \begin{itemize}
                \item Antwortformat und Inhalt sind Lehrkraft nicht bekannt
                \item Lösungsweg auch häufig nicht vorgegeben
                \item z.B. Gestaltungsaufgaben
            \end{itemize}
            \item Halboffene Aufgaben
            \begin{itemize}
                \item Antwortformat und Inhalt sind Lehrkraft bekannt
                \item Lösungsweg nicht vorgegeben
                \item z.B. Text-, Kurzantwort, Ergänzung, Zuordnungen
            \end{itemize}
            \item Geschlossene Aufgaben
            \begin{itemize}
                \item Antworten sind SuS und Lehrkraft bekannt
                \item Häufig kein Lösungsweg vorhanden
                \item z.B. Multiple Choice, Lückentexte, Zuordnungen
            \end{itemize}
        \end{itemize}
        \item Offenheit kann man auch durch den 'Weg' durch die Aufgaben definieren:
        \begin{itemize}
            \item Start: Situation der Frage, Informationen, die gegeben werden
            \item Weg: Vorgehensweise, die SuS wählen können
            \item Ziel: Ergebnis, das erreicht werden soll
        \end{itemize}
        \item Jedes dieser Punkte kann offen oder geschlossen sein, führt also zu 8 verschiedenen Aufgabenklassifikationen:
        \begin{tabular}{|P{1.5cm}|P{1.5cm}|P{1.5cm}|p{4cm}|}
            \hline
            \textbf{Start} & \textbf{Weg} & \textbf{Ziel} & \textbf{Klassifikation} \\
            \hline
            X & X & X & Beispielaufgabe \\
            \hline
            X & X & - & geschlossene Aufgabe \\
            \hline
            X & - & X & Begründungsaufgabe \\
            \hline
            X & - & - & Problemaufgabe \\
            \hline
            - & - & - & offene Situation \\
            \hline
            - & X & X & Umkehraufgabe \\
            \hline
            - & - & X & Problemumkehr \\
            \hline
            - & X & - & Anwendungssucht \\
            \hline
        \end{tabular}
    \end{itemize}
\end{block}

\begin{block}{Differenzierbarkeit von Aufgaben}
    \begin{itemize}
        \item Heterogene Vorkenntnisse, Fähigkeiten etc.
        \item 1. Schritt: Äußere Differenzierung(Lerngruppen homogener gestalten)
        \item 2. Schritt: Innere Differenzierung(u.a. durch Aufgaben)
        \item SuS steuern Auswahl der Aufgaben, die dann parallel bearbeitet werden
        \item Selbstdifferenzierende Aufgaben kümmern sich da selber drum
    \end{itemize}
\end{block}

\begin{block}{Qualitätskriterien für Aufgaben}
    \begin{itemize}
        \item Kontextorientierung
        \item Offenheit(s.o.)
        \item Differenzierbarkeit(Individualisierung)(s.o.)
        \item Aufgabentypen(siehe 'Offenheit von Aufgaben', Tabelle)
        \item Inhalts- und Prozessbereiche
        \item Nachhaltigkeit(das Ziel der Aufgabe wird Langfristig erreicht)
        \item Authentizität(s.o.)
    \end{itemize}
\end{block}

\begin{block}{Klare Handlungsanweisungen}
    \begin{itemize}
        \item SuS brauchen Sicherheit, um möglichst effizient arbeiten zu können
        \item Aufgaben sollten keine Unsicherheiten erzeugen
        \item Gutes Hilfsmittel: Operatorenliste aus dem Zentralabitur
        \item Operatoren sind sehr genau definiert $\Rightarrow$ Sicherheit
    \end{itemize}
\end{block}

\begin{block}{Prüfungsaufgaben}
    \begin{itemize}
        \item Spätestens im Abitur große Vielfalt der Aufgaben
        \item $\Rightarrow$ möglichst früh Methodenvielfalt im Unterricht
        \item Zeitaufwand soll den Punkten zugeordnet sein (mehr Zeitaufwand = mehr Punkte)
        \item Wie immer: vom einfachen zum Schwierigen
        \item Selber durchrechnen, um Probleme zu erkennen und Zeitaufwand abzuschätzen
    \end{itemize}
\end{block}

\begin{block}{Lernerfolgsüberprüfungen(In der Informatik)}
    \begin{itemize}
        \item Anders als Prüfungen nicht benotet, sondern dienen der Lernstandserhebung
        \item Indikatoren, die überprüft werden können:
        \begin{itemize}
            \item Fähigkeit, Problematiken an Aufgaben zu erkennen
            \item Strukturierung von Problemen
            \item Fähigkeit, Lösungsstrategien zu entwickeln
            \item Auswahl von Medien, Software
            \item Teillösungen testen und korrigieren
            \item Überprüfung der Lösung auf Angemessenheit
            \item Umgang mit Rechner und Maus/Tastatur
            \item Sorgfalt und Durchhaltevermögen beim durchführen von Aufgaben
            \item Fachgerechtes Äußern über eigene und andere Arbeiten
        \end{itemize}
    \end{itemize}
\end{block}

\begin{block}{Sonstige Mitarbeit}
    \begin{itemize}
        \item Unterrrichtsbeiträge
        \item Hausaufgaben
        \item Mitarbeit in Gruppen
        \item Schriftliche Übungen
        \item Referate, Protokolle
        \item Beiträge zu Projektarbeiten
        \begin{itemize}
            \item Lehrkraft verschafft sich während der Projektarbeit Überblick über Arbeitsstände, Verständnis und Leistungsvermögen der einzelnen SuS
            \item SuS tragen Berichte vor oder stellen Dokumentation bereit
            \item ggf. Gliederung und Zuweisung von Teilaufgaben unter den Gruppenmitgliedern (zur Not durch Lehrkraft)
            \item Aus den Informationen stellt die Lehrkraft die individuelle Leistung der SuS fest, mit Berücksichtigung von
            \begin{itemize}
                \item Soziale, Fachliche, Organisatotische Fähigkeiten
                \item Dokumentation
                \item Präsentation
            \end{itemize}
        \end{itemize}
    \end{itemize}
\end{block}

\begin{block}{AFBs in den Abiturprüfungen}
    \begin{itemize}
        \item AFB I: Reproduktion
        \begin{itemize}
            \item Wiedergabe von Sachverhalten, Begriffen
            \item Beschreibung und Darstellung bekannter Verfahren
            \item Verwendung bekannter Verfahren in wiederholendem Zusammenhang
            \item nennen, beschreiben, darstellen, wiedergeben, aufschreiben, definieren, beschriften
        \end{itemize}
        \item AFB II: Transfer
        \begin{itemize}
            \item Anwendung von bekannten Verfahren auf neue Probleme(aber ähnliche Probleme)
            \item Übertragung von Bekanntem auf neue Problemstellungen in bekanntem Zusammenhang
            \item erklären, analysieren, zuordnen, unterscheiden, vergleichen, interpretieren, anwenden, berechnen, einordnen, ermitteln, skizzieren
        \end{itemize}
        \item AFB III: Problemlösen und Bewerten
        \begin{itemize}
            \item Selbstständiges Bearbeiten neuer Probleme
            \item Anpassen von Verfahren/Methoden
            \item beurteilen, bewerten, erörtern, prüfen, entwickeln, gestalten, formulieren, diskutieren, reflektieren, übertragen, Stellung nehmen
        \end{itemize}
    \end{itemize}
\end{block}

\begin{block}{Mündliche Abiturprüfung(Informatik)}
    \begin{itemize}
        \item Schriftl. Prüfung: SuS zeigen, dass sie Begriffe und Methiden kennt und anwenden kann
        \item Mündl. Prüfung: SuS zeigen, ob sie die Begriffe und Methoden auch verstehen
        \item Darstellung und Begründung im Vordergrund
        \item Detaildarstellungen sind hinderlich dabei, verschiedene fachl. und meth. Kompetenzen zu prüfen(Zeitmangel)
        \item Wichtig:
        \begin{itemize}
            \item Unterricht ist da vorbei, keine Belehrungen
            \item 'Herumhacken' auf Bezeichnungen nicht erwünscht
            \item Prüfungskonzept zu starr gestalten ist hinderlich
            \item Der Fokus soll immer auf dem Prüfungsgegenstand liegen
            \item Unterbrechungen sollen möglichst vermieden werden
            \item Falls SuS über- oder unterfordert scheint, soll das Anforderungsniveau angepasst werden
        \end{itemize}
    \end{itemize}
\end{block}

\begin{block}{Naturwissenschaften im Kontext}
    Bei Naturwissenschaftl. Fächern befindet sich ein Großteil der Inhalte im nicht-sichtbaren Bereich
    \begin{itemize}
        \item Als Lösung dazu: NiK(aufgeteilt in bik(Bio), CHik(Chemie) und piko(Physik))
        \item CHik:
        \begin{itemize}
            \item Ausgehend von der Lebenswelt der SuS werden Inhalte dekontextualisiert(s.u.) und auf Basiskonzepte zurückgeführt
            \item der Unterrichtsaufbau ist Vierphasig:
            \begin{itemize}
                \item 1. Begegnung: Es werden nach Alltagssituationen gesucht, in denen die Inhalte vorkommen
                \item 2. Neugier- und Planungsphase: SuS formulieren eigene Fragen zum Phänomen, äußern Vermutungen und planen mögliche Untersuchungen. Ziel ist es, aus dem Kontext heraus ein Erkenntnisinteresse zu entwickeln.
                \item 3. Erarbeitungsphase: Chemische Konzepte, Modelle und Methoden werden gezielt erarbeitet, um die aufgeworfenen Fragen zu beantworten. Dabei kommen auch Experimente, Recherchen oder Modellbildungen zum Einsatz.
                \item 4. Vernetzungsphase: Die neu gewonnenen Erkenntnisse werden auf andere Kontexte übertragen und in die Fachsystematik eingeordnet. Ziel ist eine nachhaltige Verankerung des Wissens und die Förderung von Transferkompetenz.
            \end{itemize}
        \end{itemize}
        \item bik:
        \begin{itemize}
            \item Kontext ist Sinnstiftende Anwendung von Fachwissen
            \item Erweiterung des Unterrichtsaufbaus um eine fünfte Phase:
            \item 5. Reflexions- und Übungsphase: SuS reflektieren über die erarbeiteten Inhalte und üben deren Anwendung. Dabei werden auch Verbindungen zu anderen Fächern und Alltagskontexten hergestellt.
        \end{itemize}
        \item piko:
        \begin{itemize}
            \item Kontexte können aus dem Thema, der Lernumgebung oder der Lebenswelt der SuS stammen
        \end{itemize}
        \item Alltagserfahrungen sollen Teil des Unterrichts sein(nicht nur der Einstieg)
        \item Vorstellungen der SuS sollen berücksichtigt werden
        \item Unterricht wird an den Lernbedürfnissen und Erfahrungen der SuS orientiert
        \item $\Rightarrow$ nicht an der fachlichen Logik
        \item Kompetenzaufbau kommt von SuS selber(s.o.: Konstruktivismus)
        \item Kritik an NiK:
        \begin{itemize}
            \item Fachsystematik geht verloren
            \item die KMK-Standards werden nicht komplett abgedeckt
            \item Kontexte sehr komplex $\Rightarrow$ Dekontextualisierung sehr großer Teil des Zeitaufwands
            \item Alltagsbezug ist keine neue Idee
        \end{itemize}
    \end{itemize}
\end{block}

\begin{block}{IniK}
    \begin{itemize}
        \item Orientierung an für SuS bedeutsamen Kontexten
        \item Orientierung an Empfehlungen der GI für Bildungsstandards
        \item Methoden, die aktivieren und kooperativ sind
        \item Kontexte in der IniK:
        \begin{itemize}
            \item Lebensweltbezug, evtl. erlebbar für SuS
            \item Zeitstabil, also nicht an aktuellem Stand der Technik orientiert
            \item informatische Inhalte spielen zentrale Rolle im Kontext
            \item Kontext ist gesellschaftl. relevant(Breite)
            \item Relevanz der informatischen Inhalte für die informatische Bildung(Tiefe)
            \item Öffnet zusätzlich Blick für neue Inhaltsbereiche
            \item die informatischen Inhalte sind auf Schulniveau reduzierbar
            \item Übertragbarkeit auf andere Kontexte möglich(Transfer)
            \item Beispiele:
            \begin{itemize}
                \item Chatbots
                \item Emails
                \item RFID
            \end{itemize}
        \end{itemize}
    \end{itemize}
\end{block}

\begin{block}{Dekontextualisierung}
    Der Kontext aus der Erfahrung der SuS wird auf die Basiskonzepte des Faches reduziert
\end{block}

\begin{block}{Phänomene}
    \begin{itemize}
        \item Die konkreten Erfahrungen der SuS mit einem Kontext
        \item Drei Arten von Phänomen(in der Informatik):
        \begin{itemize}
            \item 1. Direkt mit Informatiksystemen verbunden
            \item 2. Indirekt mit Informatiksystemen verbunden
            \item 3. Nicht direkt mit Informatiksystemen verbunden, aber inhärent informatisch
        \end{itemize}
        \item Wenn Kontexte von den SuS nicht wahrgenommen werden, werden ggf. Medien zur Erschließung notwendig
    \end{itemize}
\end{block}

\end{document}